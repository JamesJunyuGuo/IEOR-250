%% LyX 2.2.2 created this file.  For more info, see http://www.lyx.org/.
%% Do not edit unless you really know what you are doing.
%%%%%%%%%%%%%%%%%%%%%%%%%%%%%% User specified LaTeX commands.
%\usepackage{multirow}
%\usepackage{floatrow}

\documentclass[11pt]{article}%
\usepackage[latin9]{inputenc}
\usepackage{amsmath}
\usepackage{amssymb}
\usepackage[authoryear]{natbib}
\usepackage{float}
\usepackage{graphicx}
\usepackage{amsfonts}
\usepackage{amsthm}
\usepackage{enumerate}
\usepackage{epsfig}
\usepackage{ifthen}
\usepackage{latexsym}
\usepackage{syntonly}
\usepackage{rotating}
\usepackage{lscape}
\usepackage{color}
\usepackage{booktabs}
\usepackage[bottom]{footmisc}
\usepackage{epstopdf}
\usepackage{makeidx}
\usepackage{xr}
% \usepackage{refcheck}%
\setcounter{MaxMatrixCols}{30}
%TCIDATA{OutputFilter=latex2.dll}
%TCIDATA{Version=5.50.0.2960}
%TCIDATA{CSTFile=LaTeX article (bright).cst}
%TCIDATA{LastRevised=Wednesday, August 16, 2023 19:55:35}
%TCIDATA{<META NAME="GraphicsSave" CONTENT="32">}
%TCIDATA{<META NAME="SaveForMode" CONTENT="1">}
%TCIDATA{BibliographyScheme=BibTeX}
%TCIDATA{Language=American English}
%BeginMSIPreambleData
\providecommand{\U}[1]{\protect\rule{.1in}{.1in}}
%EndMSIPreambleData
\textwidth=6.6in
\textheight=8.9in
\headheight=0.0in
\oddsidemargin=0.0in
\headsep=0.0in
\topmargin=0.0in
\newtheorem{theorem}{Theorem}
\newtheorem{corollary}{Corollary}
\newtheorem{case}{Case}
\newtheorem{lemma}{Lemma}
\newtheorem{proposition}{Proposition}
\newtheorem{assumption}{Assumption}
\theoremstyle{definition}
\newtheorem{definition}{Definition}
\newtheorem{example}{Example}
\newtheorem{remark}{Remark}
\def\baselinestretch{1.3}
\newcommand{\abs}[1]{\lvert#1\rvert}
\newcommand{\norm}[1]{\left\lVert#1\right\rVert}
\DeclareMathOperator*{\adjugate}{adj}
\DeclareMathOperator*{\sign}{sgn}
\allowdisplaybreaks
% \externaldocument{Wealth_effect_HARA_Supp_2022}






\begin{document}

\title{INDENG 250 PS1}
\author{Junyu Guo}
\date{\today }
\maketitle

1. Supply chain Management Definition.    
It is the set of practices required to perform the functions of a 
Supply chain and to make them more efficient, profitable, equitable, sustainable
less costly, less wasteful and less stressful.


2. Write out $D_t=\cdots$.    

2.1 Under the simple average model, we make the assumption that
\begin{equation}
    D_t = I + \epsilon_t.
    \label{2.1}
\end{equation}   

2.2 Under the moving average model, we have    
\begin{equation}
    D_t = I_t + \epsilon_t.
\end{equation}

2.3 Under the exponential model, we have    
\begin{equation}
     D_t = I_t + \epsilon_t.
\end{equation}




2.4 Under the double exponential smoothing model, we assume
\begin{equation}
    D_t= I+tS + \epsilon_t.
\end{equation}    

2.5 Under  triple exponential smoothing model, we assume
\begin{equation}
    D_t = (I+tS)c_t + \epsilon_t, \text{where }    \sum_{t}c_t = N.
\end{equation}      

3. Insert all the figures generated by the code.     


\begin{figure}[H]
    \centering
    \includegraphics*[scale = 0.5]{decomposition.pdf}
    \caption{decomposition}
    \label{fig: data0}
\end{figure}

\begin{figure}[H]
    \centering
    \includegraphics*[scale = 0.3]{historical.pdf}
    \caption{historical}
    \label{fig: data1}
\end{figure}
\begin{figure}[H]
    \centering
    \includegraphics*[scale = 0.25]{SES.pdf}
    \caption{SES}
    \label{fig: data2}
\end{figure}

\begin{figure}[H]
    \centering
    \includegraphics*[scale = 0.25]{DES.pdf}
    \caption{DES}
    \label{fig: data3}
\end{figure}

\begin{figure}[H]
    \centering
    \includegraphics*[scale = 0.25]{TEs.pdf}
    \caption{DES}
    \label{fig: data4}
\end{figure}   


4. Now we show that the simple average is the 
least square minimizer. Assume $\sigma_t\sim \mathcal{N}(0,\sigma^2)$.   
Suppose we would like to have a parmeter $\theta$, with the estimator as 
$\hat{x}_t = \sum_{i=0}^{t-1}\theta_{i}x_i$. Now we would like to minimize the 
square loss as 
\begin{equation}
    \mathcal{L} = \mathbb{E}[\left(x_t-\hat{x}_t\right)^2].
    \label{loss}
\end{equation}
Also we denote $\theta = \left(\theta_0,\cdots,\theta_{t-1}\right)^{\top}$, and 
we can simplify (\ref{loss}) into 
\begin{equation}
    \theta^{\top}A \theta - b^{\top}\theta + C, 
\end{equation}
where $C= \mathbb{E}[x_t]^2$, $A = (\mathbb{E}[x_i x_j])_{t\times t}$, and $b = (\mathbb{E}[x_i x_t])_{0\leq i\leq t-1}$. Since this is a convex function with respect to $\theta$, then we set the derivatives be $0$ can we can solve $\theta^*$, which is the optimal solution for linear approximation.     
Using the definition, assume $\mathbb{E}[x_0]^2 = a^2$, then $A$ can be written into 
\begin{equation}
    A = \begin{pmatrix}
    a^2+\sigma^2 & a^2 &\cdots & a^2\\
    a^2 & a^2+ \sigma^2 &\cdots& a^2\\
    \cdots & \cdots & & \cdots\\
    a^2 & a^2 &\cdots& a^2 + \sigma^2\\
\end{pmatrix}, b= \begin{pmatrix}
    a^2\\
    a^2\\
    \cdots\\
    a^2\\
\end{pmatrix}.
\end{equation}
Using the definition of the matrix, we can have that 
\begin{equation}
    A\theta^{*} = b. 
\end{equation}
It's easy to obtain that $\theta_0= \theta_1=\cdots= \theta_{t-1}$. Then $\theta_i = \frac{a^2}{ta^2+\sigma^2}\approx \frac{1}{t}$. When $t$ is sufficiently large, we can say that $\theta_i = \frac{1}{t}$ is the best coefficient for approximating $x_t$. 
\end{document}